T.\+D.\+A. \hyperlink{classCronologia}{Cronologia}

Una instancia {\itshape c} del tipo de dato abstracto {\ttfamily \hyperlink{classCronologia}{Cronologia}} es un objeto del conjunto de fechas históricas compuestos por un valor entero que representa el numero de fechas, otro que representa el tamaño del vector dinamico utilizado y el vector dinamico de fechas. Un ejemplo de su uso\+: 
\begin{DoxyCodeInclude}
\textcolor{preprocessor}{#include "\hyperlink{cronologia_8h}{cronologia.h}"}
\textcolor{preprocessor}{#include <fstream>}
\textcolor{preprocessor}{#include <iostream>}

\textcolor{keyword}{using namespace }\hyperlink{namespacestd}{std};

\textcolor{keywordtype}{int} main(\textcolor{keywordtype}{int} argc, \textcolor{keywordtype}{char} * argv[])\{



  \textcolor{keywordflow}{if} (argc!=3)\{
      cout<<\textcolor{stringliteral}{"Dime el nombre del fichero con la cronologia"}<<endl;
      \textcolor{keywordflow}{return} 0;
   \}

   ifstream f (argv[1]);
   \textcolor{keywordflow}{if} (!f)\{
    cout<<\textcolor{stringliteral}{"No puedo abrir el fichero "}<<argv[1]<<endl;
    \textcolor{keywordflow}{return} 0;
   \}
   
   \hyperlink{classCronologia}{Cronologia} mi\_cronologia;
   f>>mi\_cronologia; \textcolor{comment}{//Cargamos en memoria la cronología.}


   \textcolor{comment}{/* Exhibir aquí la funcionalidad programada para el TDA Cronologia / TDA FechaHistorica */} 

   \textcolor{comment}{// Algunas sugerencias: }
   \textcolor{comment}{// - Obtener los eventos acaecidos en un año dado}
   \textcolor{comment}{// - Obtener la (sub)cronología de eventos históricos acaecidos en [anioDesde, anioHasta], donde
       anioDesde y anioHasta son proporcionados por el usuario}
   \textcolor{comment}{// - Obtener la (sub)cronología de eventos históricos asociados a una palabra clave. Ejemplo: la
       cronología de eventos asociados a "Einstein"}
   \textcolor{comment}{// - Operadores sobre cronologías, ejemplo: Unión de dos cronologías (la cronología resultante debería
       estar ordenada)}
   \textcolor{comment}{// - Cualquier otra funcionalidad que consideréis de interés}
   cout << mi\_cronologia ;
   f.close();
   ofstream s (argv[2]);
   \textcolor{keywordflow}{if} (!s)\{
    cout<<\textcolor{stringliteral}{"No puedo abrir el fichero de salida "}<<argv[2]<<endl;
    \textcolor{keywordflow}{return} 0;
   \}
   s << mi\_cronologia;
   s.close();
   ifstream w (argv[2]);
   \textcolor{keywordflow}{if} (!w)\{
    cout<<\textcolor{stringliteral}{"No puedo abrir el fichero "}<<argv[2]<<endl;
    \textcolor{keywordflow}{return} 0;
   \}
   \hyperlink{classFechaHistorica}{FechaHistorica} mifecha;
   w >> mifecha;
   w.close();
   \hyperlink{classCronologia}{Cronologia} mi\_cronologia2;
   w.open(argv[2]);
   w >> mi\_cronologia2;
   cout << mi\_cronologia2 << mifecha;
   w.close();
\}
\end{DoxyCodeInclude}


\begin{DoxyAuthor}{Autor}
Emilio José Hoyo Medina 

Stefan Parvanov 
\end{DoxyAuthor}
\begin{DoxyDate}{Fecha}
Octubre 2017
\end{DoxyDate}
\hypertarget{repBConjunto_invConjunto}{}\section{Invariante de la representación}\label{repBConjunto_invConjunto}
El invariante es {\itshape rep.\+num\+Fechas$>$=0} rep.\+tam\+Fechas$>$=0\hypertarget{repBConjunto_faConjunto}{}\section{Función de abstracción}\label{repBConjunto_faConjunto}
Un objeto valido {\itshape rep} del T\+DA \hyperlink{classCronologia}{Cronologia} representa a un conjunto de fechas historicas de cardinal rep.\+num\+Fechas( almacenadas en el vector dinamico rep.\+fechas de tamaño rep.\+tam\+Fechas). 