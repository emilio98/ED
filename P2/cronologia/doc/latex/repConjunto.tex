T.\+D.\+A. \hyperlink{classFechaHistorica}{Fecha\+Historica}

Una instancia {\itshape c} del tipo de dato abstracto {\ttfamily \hyperlink{classFechaHistorica}{Fecha\+Historica}} es un objeto del conjunto de los acontecimientos históricos en una fecha, compuestos por un valor entero que representa el año y un conjunto de cadenas de caracteres (strings) que representan los acontecimientos acaecidos ese año. Un ejemplo de su uso\+: 
\begin{DoxyCodeInclude}

\textcolor{preprocessor}{#include "cronologia.h"}
\textcolor{preprocessor}{#include <fstream>}
\textcolor{preprocessor}{#include <iostream>}

\textcolor{keyword}{using namespace }\hyperlink{namespacestd}{std};

\textcolor{keywordtype}{int} main(\textcolor{keywordtype}{int} argc, \textcolor{keywordtype}{char} * argv[])\{



  \textcolor{keywordflow}{if} (argc!=3)\{
      cout<<\textcolor{stringliteral}{"Dime el nombre del fichero con la cronologia"}<<endl;
      \textcolor{keywordflow}{return} 0;
   \}

   ifstream f (argv[1]);
   \textcolor{keywordflow}{if} (!f)\{
    cout<<\textcolor{stringliteral}{"No puedo abrir el fichero "}<<argv[1]<<endl;
    \textcolor{keywordflow}{return} 0;
   \}
   
   Cronologia mi\_cronologia;
   f>>mi\_cronologia; \textcolor{comment}{//Cargamos en memoria la cronología.}


   \textcolor{comment}{/* Exhibir aquí la funcionalidad programada para el TDA Cronologia / TDA FechaHistorica */} 

   \textcolor{comment}{// Algunas sugerencias: }
   \textcolor{comment}{// - Obtener los eventos acaecidos en un año dado}
   \textcolor{comment}{// - Obtener la (sub)cronología de eventos históricos acaecidos en [anioDesde, anioHasta], donde
       anioDesde y anioHasta son proporcionados por el usuario}
   \textcolor{comment}{// - Obtener la (sub)cronología de eventos históricos asociados a una palabra clave. Ejemplo: la
       cronología de eventos asociados a "Einstein"}
   \textcolor{comment}{// - Operadores sobre cronologías, ejemplo: Unión de dos cronologías (la cronología resultante debería
       estar ordenada)}
   \textcolor{comment}{// - Cualquier otra funcionalidad que consideréis de interés}
   cout << mi\_cronologia ;
   f.close();
   ofstream s (argv[2]);
   \textcolor{keywordflow}{if} (!s)\{
    cout<<\textcolor{stringliteral}{"No puedo abrir el fichero de salida "}<<argv[2]<<endl;
    \textcolor{keywordflow}{return} 0;
   \}
   s << mi\_cronologia;
   s.close();
   ifstream w (argv[2]);
   \textcolor{keywordflow}{if} (!w)\{
    cout<<\textcolor{stringliteral}{"No puedo abrir el fichero "}<<argv[2]<<endl;
    \textcolor{keywordflow}{return} 0;
   \}
   \hyperlink{classFechaHistorica}{FechaHistorica} mifecha;
   w >> mifecha;
   w.close();
   Cronologia mi\_cronologia2;
   w.open(argv[2]);
   w >> mi\_cronologia2;
   cout << mi\_cronologia2 << mifecha;
   w.close();
\}
\end{DoxyCodeInclude}


\begin{DoxyAuthor}{Autor}
Emilio José Hoyo Medina 

Stefan Parvanov 
\end{DoxyAuthor}
\begin{DoxyDate}{Fecha}
Octubre 2017
\end{DoxyDate}
\hypertarget{repConjunto_invConjunto}{}\section{Invariante de la representación}\label{repConjunto_invConjunto}
El invariante es {\itshape rep.\+año$<$=2018} rep.\+numsucesos$>$=0 rep.\+tamstring$>$=0\hypertarget{repConjunto_faConjunto}{}\section{Función de abstracción}\label{repConjunto_faConjunto}
Un objeto válido {\itshape rep} del T\+DA \hyperlink{classFechaHistorica}{Fecha\+Historica} representa a un conjunto de sucesos históricos de cardinal rep.\+numsucesos (almacenados en el vector dinámico de strings rep.\+sucesos, de tamaño rep.\+tamstring) y acaecidos en el año rep.\+anio. 