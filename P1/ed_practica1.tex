\documentclass{article}
\usepackage{listings}

\begin{document}

\lstset{language=C++}
\lstset{numbers=left}

\title{Práctica 1}
\author{Emilio Jos\`e Hoyo \\ Stefan Parvanov}
\maketitle

\section{Ejercicio 1 : Ordenación de la burbuja}
El siguiente código realiza la ordenación mediante el algoritmo de la burbuja:\\
\begin{lstlisting}
      void ordenar(int *v, int n) {
        for (int i=0; i<n-1; i++)
          for (int j=0; j<n-i-1; j++)
            if (v[j]>v[j+1]) {
              int aux = v[j];
              v[j] = v[j+1];
              v[j+1] = aux;
		} 
      }
\end{lstlisting}

Calcule la eficiencia teórica de este algoritmo. A continuación replique el experimento que se ha hecho antes (búsqueda lineal) con este nuevo código. Debe:
\begin{itemize}
	\item Crear un fichero ordenacion.cpp con el programa completo para realizar una ejecución del algoritmo.
	\item Crear un script ejecuciones\_ordenacion.csh en C-Shell que permite ejecutar varias veces el programa anterior y generar un fichero con los datos obtenidos.
	\item Usar gnuplot para dibujar los datos obtenidos en el apartado previo
\end{itemize}
Los datos deben contener tiempos de ejecución para tamaños del vector 100, 600, 1100, ...,30000. \\
Pruebe a dibujar superpuestas la función con la eficiencia teórica y la empírica. ¿Qu\`e sucede? 
\clearpage

Empezamos calculando la eficiencia te\`orica:
\begin{itemize}
	\item Linea 2: \\ Al principio del bucle siempre tendremos 3 OE :la asignaci\`on i=0\ ; la resta  n-1 y la comparaci\`on. Al final de cada iteraci\`o tambi\`en se ejecuta 1 OE (el incremento del contador). El bucle ejecuta n-2 veces su cuerpo.
	\item Linea 3: \\ Al igual que antes tenemos las 3 OE iniciales y 1 OE en cada iteraci\`on. El bucle se ejecuta n-i-2 veces.
	\item Linea 4: \\ Se ejecutan 4 OE : 2 accesos a memoria, una operaci\`on aritm\`etica y una comparaci\`on.
	\item Lineas 5, 6, y 7 \\  Se ejecutan 9 OE.
\end{itemize}
	Obtenemos el siguiente tiempo de ejecuci\`on: 
	\begin{equation}
		T(n)= 3 + \sum\limits_{i=1}^{n-1}{3+1+3+\sum\limits_{j=1}^{n-i-1}{3+1+4+9}}
	\end{equation}
	Mediante gnuplot
	
\clearpage
\section{Ejercicio 2 : Ajuste en la ordenación de la burbuja}
Replique el experimento de ajuste por regresión a los resultados obtenidos en el ejercicio 1 que calculaba la eficiencia del algoritmo de ordenación de la burbuja. Para ello considere que f(x) es de la forma ax2+bx+c

\clearpage
\section{Ejercicio 3 : Problemas de precisión}
Junto con este guión se le ha suministrado un fichero ejercicio\_desc.cpp. En él se ha implementado un algoritmo. Se pide que:
\begin{itemize}
	\item Explique qué hace este algoritmo.
	\item Calcule su eficiencia teórica.
	\item Calcule su eficiencia empírica.
\end{itemize}
Si visualiza la eficiencia empírica debería notar algo anormal. Explíquelo y proponga una solución. Compruebe que su solución es correcta. Una vez resuelto el problema realice la regresión para ajustar la curva teórica a la empírica.


\clearpage
\section{Ejercicio 4 :}
Retome el ejercicio de ordenación mediante el algoritmo de la burbuja. Debe modificar el
código que genera los datos de entrada para situarnos en dos escenarios diferentes:
\begin{itemize}
	\item El mejor caso posible. Para este algoritmo, si la entrada es un vector que ya está ordenado el tiempo de cómputo es menor ya que no tiene que intercambiar ningún elemento
	\item El peor caso posible. Si la entrada es un vector ordenado en orden inverso estaremos en la peor situación posible ya que en cada iteración del bucle interno hay que hacer un intercambio.
\end{itemize}
	Calcule la eficiencia empírica en ambos escenarios y compárela con el resultado del ejercicio 1.
\end{document}